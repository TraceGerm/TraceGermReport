\chapter{Περιγραφή περιπτώσεων χρήσης}

\section{Εισαγωγή}
Σε αυτό το κεφάλαιο θα αναλύσουμε τις περιπτώσεις χρήσης του συστήματος. Θα αποφύγουμε να εισέλθουμε σε λεπτομέρειες του σχεδιασμού και της υλοποίησης του συστήματος καθώς υπάρχει κεφάλαιο παρακάτω αφιερωμένο σε αυτό το σκοπό.

\section{Εφαρμογή κινητών συσκευών}

\subsection{Σενάρια χρήσης}
Ο χρήστης θα πρέπει να μπορεί να χρησιμοποιήσει την εφαρμογή ανεξαρτήτως πλατφόρμας κινητής συσκευής. Παρακάτω θα παρουσιαστούν οι διάφορες περιπτώσεις χρήσης της εφαρμογής σε μορφή πίνακα \citep{uml}.

%--------------------- USE-CASE 1

\begin{table}[h]
 \caption{Δημιουργία νέου λογαριασμού}
\begin{center}
\begin{tabular}{ | m{10em} |  m{25em} | } 
\hline
 Τίτλος & Δημιουργία νέου λογαριασμού \\ 
\hline
 Περιγραφή & Ο χρήστης θα μπορεί να δημιουργήσει ένα προσωπικό λογαριασμό για να μπορέσει να χρησημοποιήσει την εφαρμογή. \\ 
\hline
 Επιτυχές σενάριο &  Ο χρήστης δημιουργεί ένα νέο προσωπικό λογαριασμό.\\
\hline
 Ανεπιτυχές σενάριο  & Ο χρήστης αποτυγχάνει καθώς το usename είναι καταχωρυμένο σε άλλο χρήστη. \\ 
\hline
 Βασικοί χρήστες  & Απλός χρήστης \\ 
\hline
 Βασική ροή  & 
\begin{enumerate}
\item Η εφαρμογή εμφανίζει την φόρμα εγγραφής
\item Ο χρήστης συμπληρώνει τα απαραίτητα πεδία
\item Ο χρήστης αποστέλλει τα πεδία
\item Το σύστημα ελέγχει τη ορθότητα των δεδομένων και δημιουργεί το νέο λογαριασμό χρήστη
\end{enumerate}
 \\ 
\hline
 Εναλλακτικό σενάριο  & Αποτυχία δημιουργίας λογαριασμού  \\ 
\hline
\end{tabular}
\end{center}
\end{table}

%--------------------- USE-CASE 2

\begin{table}[h]
 \caption{Είσοδος του χρήστη στην εφαρμογή}
\begin{center}
\begin{tabular}{ | m{10em} |  m{25em} | } 
\hline
 Τίτλος & Είσοδος του χρήστη στην εφαρμογή \\ 
\hline
 Περιγραφή & Ο χρήστης θα μπορεί να εισέρχεται στην εφαρμογή κάνοντας χρήση του username και password. \\ 
\hline
 Επιτυχές σενάριο & Ο χρήστης εισέρχεται με επιτυχία στην εφαρμογή.\\
\hline
 Ανεπιτυχές σενάριο  & Ο χρήστης  αποτυγχάνει καθώς τα διαπιστευτήριά του είναι λανθασμένα. \\ 
\hline
 Βασικοί χρήστες  & Απλός χρήστης \\ 
\hline
 Βασική ροή  & 
\begin{enumerate}
\item Η εφαρμογή εμφανίζει την φόρμα εισόδου
\item Ο χρήστης συμπληρώνει τα απαραίτητα πεδία
\item Ο χρήστης αποστέλλει τα πεδία
\item Το σύστημα ελέγχει τη ορθότητα των δεδομένων και επιτρέπει την είσοδο του χρήστη
\end{enumerate}
 \\ 
\hline
 Εναλλακτικό σενάριο  & Αποτυχία εισόδου του χρήστη  \\ 
\hline
\end{tabular}
\end{center}
\end{table}

%--------------------- USE-CASE 3

\begin{table}[h]
 \caption{Ενεργοποίηση της καταγραφής των τοποθεσιών του χρήστη.}
\begin{center}
\begin{tabular}{ | m{10em} |  m{25em} | } 
\hline
 Τίτλος & Ενεργοποίηση της καταγραφής των τοποθεσιών του χρήστη. \\ 
\hline
 Περιγραφή & Ο χρήστης θα μπορεί να ενεργοποιήσει την καταγραφή των τοποθεσιών του. \\ 
\hline
 Επιτυχές σενάριο & Ο χρήστης ενεργοποιεί με επιτυχία την καταγραφή.\\
\hline
 Ανεπιτυχές σενάριο  & Η συσκευή αποτυγχάνει να εκκινήσει την υπηρεσία καταγραφής. \\ 
\hline
 Βασικοί χρήστες  & Απλός χρήστης \\ 
\hline
 Βασική ροή  & 
\begin{enumerate}
\item Η εφαρμογή εμφανίζει την σελίδα καταγραφής τοποθεσιών
\item Ο χρήστης ενεργοποιεί την καταγραφή
\item Η υπηρεσία εκκινεί την καταγραφή στο πίσω μέρος της εφαρμογής
\end{enumerate}
 \\ 
\hline
 Εναλλακτικό σενάριο  & Αποτυχία εκκίνησης της υπηρεσίας  \\ 
\hline
\end{tabular}
\end{center}
\end{table}

%--------------------- USE-CASE 4

\begin{table}[h]
 \caption{Εμφάνιση των καταγεγραμμένων τοποθεσιών του χρήστη}
\begin{center}
\begin{tabular}{ | m{10em} |  m{25em} | } 
\hline
 Τίτλος & Εμφάνιση των καταγεγραμμένων τοποθεσιών του χρήστη. \\ 
\hline
 Περιγραφή & Ο χρήστης θα μπορεί να εμφανίζει τις καταγεγραμμένες τοποθεσίες του σε ένα χάρτη. \\ 
\hline
 Επιτυχές σενάριο & Ο χρήστης βλέπει τις τοποθεσίες του να εμφανίζονται στο χάρτη.\\
\hline
 Ανεπιτυχές σενάριο  & Η εφαρμογή αποτυγχάνει να ανακαλέσει τις τοποθεσίες του χρήστη. \\ 
\hline
 Βασικοί χρήστες  & Απλός χρήστης \\ 
\hline
 Βασική ροή  & 
\begin{enumerate}
\item Ο χρήστης επιλέγει την σελίδα εμφάνισης των τοποθεσιών από το μενού επιλογών της εφαρμογής.
\item Η εφαρμογή φορτώνει τις τοποθεσίες του χρήστη.
\item Γίνεται εμφάνιση των τοποθεσιών στο χάρτη
\end{enumerate}
 \\ 
\hline
 Εναλλακτικό σενάριο  & Αποτυχία εύρεσης τοποθεσιών και εμφάνιση άδειου χάρτη \\ 
\hline
\end{tabular}
\end{center}
\end{table}



\begin{table}[h]
 \caption{Αποστολή προειδοποίησης από το χρήστη σε άλλους χρήστες.}
\begin{center}
\begin{tabular}{ | m{10em} |  m{25em} | } 
\hline
 Τίτλος & Αποστολή προειδοποίησης από το χρήστη σε άλλους χρήστες. \\ 
\hline
 Περιγραφή & Ο χρήστης θα έχει τη δυνατότητα αποστολής προειδοποιήσεων σε άλλους χρήστες σε περίπτωση που είναι φορέας κάποιας ασθένειας. \\ 
\hline
 Επιτυχές σενάριο & Ο χρήστης αποστέλλει με επιτυχία προειδοποίηση στους άλλους χρήστες.\\
\hline
 Ανεπιτυχές σενάριο  & Η εφαρμογή αποτυγχάνει να αποστείλει την προειδοποίηση του χρήστη. \\ 
\hline
 Βασικοί χρήστες  & Απλός χρήστης \\ 
\hline
 Βασική ροή  & 
\begin{enumerate}
\item Ο χρήστης επιλέγει την σελίδα προειδοποιήσεων από το μενού επιλογών της εφαρμογής.
\item Ο χρήστης κάνει κλικ στο κουμπί δημιουργίας νέας προειδοποίησης.
\item Η συσκεύη στέλνει στον εξυπηρετητή την προειδοποίηση του χρήστη.
\end{enumerate}
 \\ 
\hline
 Εναλλακτικό σενάριο  & Αποτυχία αποστολής της προειδοποίησης στον εξυπηρετητή.\\ 
\hline
\end{tabular}
\end{center}
\end{table}

%--------------------- USE-CASE 6

\begin{table}[h]
 \caption{Ειδοποίηση του χρήστη σε περίπτωση επαφής με κάποιο χρήστη φορέα.}
\begin{center}
\begin{tabular}{ | m{10em} |  m{25em} | } 
\hline
 Τίτλος & Ειδοποίηση του χρήστη σε περίπτωση επαφής με κάποιο χρήστη φορέα. \\ 
\hline
 Περιγραφή & Ο χρήστης θα έχει τη δυνατότητα ενεργοποίησης της υπηρεσίας παρακολούθησης ειδοποιήσεων από άλλους χρήστες φορείς κάποιας ασθένειας. \\ 
\hline
 Επιτυχές σενάριο & Ο χρήστης θα ενεργοποιεί με επιτυχία την υπηρεσία παρακολούθησης ειδοποιήσεων από άλλους χρήστες.\\
\hline
 Ανεπιτυχές σενάριο  & Η εφαρμογή αποτυγχάνει να εκκινήσει την υπηρεσία. \\ 
\hline
 Βασικοί χρήστες  & Απλός χρήστης \\ 
\hline
 Βασική ροή  & 
\begin{enumerate}
\item Ο χρήστης επιλέγει την σελίδα προειδοποιήσεων από το μενού επιλογών της εφαρμογής.
\item Ο χρήστης  θα μπορεί να ενεργοποιήσει την υπηρεσία παρακολούθησης ειδοποιήσεων από τις υπηρεσίες υγείας.
\item Η συσκεύη συνδέετε με τον εξυπηρετητή και αναμένει ειδοποιήσεις.
\end{enumerate}
 \\ 
\hline
 Εναλλακτικό σενάριο  & Αποτυχία σύνδεσης με τον εξυπηρετητή.\\ 
\hline
\end{tabular}
\end{center}
\end{table}

\clearpage
\section{Διαδικτυακή εφαρμογή}

\subsection{Σενάρια χρήσης}
Οι υπηρεσίες υγείας θα πρέπει να έχουν πρόσβαση μέσω μιας άλλης διεπαφής χρήστη στις προειδοποιήσει που έχουν αποσταλεί από τους χρήστες. Παρακάτω θα παρουσιαστούν οι διάφορες περιπτώσεις χρήσης της εφαρμογής σε μορφή πίνακα.


%--------------------- USE-CASE 1

\begin{table}[h]
 \caption{Είσοδος των υπηρεσιών υγείας στη εφαρμογή}
\begin{center}
\begin{tabular}{ | m{10em} |  m{25em} | } 
\hline
 Τίτλος & Είσοδος των υπηρεσιών υγείας στη εφαρμογή \\ 
\hline
 Περιγραφή & Οι υπηρεσίες υγείας θα μπορούν να εισέρχονται στην εφαρμογή κάνοντας χρήση προκαθορισμένου όνομα χρήστη και κωδικού.\\ 
\hline
 Επιτυχές σενάριο & Οι υπηρεσίες υγείας εισέρχονται με επιτυχία στην εφαρμογή.\\
\hline
 Ανεπιτυχές σενάριο  & Οι υπηρεσίες υγείας αποτυγχάνουν καθώς τα διαπιστευτήριά τους είναι λανθασμένα. \\ 
\hline
 Βασικοί χρήστες  & Υπηρεσίες υγείας \\ 
\hline
 Βασική ροή  & 
\begin{enumerate}
\item Η εφαρμογή εμφανίζει την φόρμα εισόδου
\item Οι υπηρεσίες υγείας συμπληρώνουν τα απαραίτητα πεδία
\item Οι υπηρεσίες υγείας αποστέλλουν τα πεδία
\item Το σύστημα ελέγχει τη ορθότητα των δεδομένων και επιτρέπει την είσοδο των υπηρεσιών υγείας
\end{enumerate}
 \\ 
\hline
 Εναλλακτικό σενάριο  & Αποτυχία εισόδου στην εφαρμογή \\ 
\hline
\end{tabular}
\end{center}
\end{table}

%--------------------- USE-CASE 2

\begin{table}[h]
 \caption{Εμφάνιση απεσταλμένων ειδοποιήσεων από τους χρήστες}
\begin{center}
\begin{tabular}{ | m{10em} |  m{25em} | } 
\hline
 Τίτλος & Εμφάνιση απεσταλμένων ειδοποιήσεων από τους χρήστες \\ 
\hline
 Περιγραφή & Οι υπηρεσίες υγείας θα μπορούν να παρακολουθούν σε πραγματικό χρόνο τις απεσταλμένες ειδοποιήσεις χρηστών.\\ 
\hline
 Επιτυχές σενάριο & Οι υπηρεσίες υγείας ενεργοποιούν την υπηρεσία παρακολούθησης ειδοποιήσεων των χρηστών.\\
\hline
 Ανεπιτυχές σενάριο  & Η εφαρμογή αποτυγχάνει να εκκινήσει την υπηρεσία. \\ 
\hline
 Βασικοί χρήστες  & Υπηρεσίες υγείας \\ 
\hline
 Βασική ροή  & 
\begin{enumerate}
\item Οι υπηρεσίες υγείας επιλέγουν την σελίδα ειδοποιήσεων από το μενού επιλογών της εφαρμογής.
\item Η συσκεύη συνδέετε με τον εξυπηρετητή και αναμένει ειδοποιήσεις.
\end{enumerate}
 \\ 
\hline
 Εναλλακτικό σενάριο  & Αποτυχία σύνδεσης με τον εξυπηρετητή \\ 
\hline
\end{tabular}
\end{center}
\end{table}

%--------------------- USE-CASE 3

\begin{table}[h]
 \caption{Αποδοχή ή μη μίας απεσταλμένης ειδοποίησης από έναν χρήστη}
\begin{center}
\begin{tabular}{ | m{10em} |  m{25em} | } 
\hline
 Τίτλος & Αποδοχή ή μη μίας απεσταλμένης ειδοποίησης από έναν χρήστη \\ 
\hline
 Περιγραφή & Οι υπηρεσίες υγείας θα μπορούν να μελετήσουν μια απεσταλμένη ειδοποίηση από κάποιον χρήστη και να την αποδεχτούν ή όχι.\\
\hline
 Επιτυχές σενάριο & Οι υπηρεσίες αποδέχονται ή μη με επιτυχία μια απεσταλμένη ειδοποίηση.\\
\hline
 Ανεπιτυχές σενάριο  & Η εφαρμογή αποτυγχάνει να αποστείλει την επιλογή των υπηρεσιών υγείας στον εξυπηρετητή. \\ 
\hline
 Βασικοί χρήστες  & Υπηρεσίες υγείας \\ 
\hline
 Βασική ροή  & 
\begin{enumerate}
\item Οι υπηρεσίες υγείας επιλέγουν μια ειδοποίηση από τη σελίδα ειδοποιήσεων
\item Η εφαρμογή εμφανίζει τις πληροφορίες του χρήστη και της ειδοποίησης
\item Οι υπηρεσίες υγείας διαλέγουν αν αποδέχονται ή όχι την ειδοποίηση κάνοντας κλικ στο αντίστοιχο κουμπί.
\end{enumerate}
 \\ 
\hline
 Εναλλακτικό σενάριο  & Αποτυχία σύνδεσης με τον εξυπηρετητή. \\ 
\hline
\end{tabular}
\end{center}
\end{table}

