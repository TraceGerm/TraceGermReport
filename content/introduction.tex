\chapter{Εισαγωγή}
\leftmark\rightmark
\section{Γενικά}
Σε περίπτωση κρουσμάτων μιας μεταδιδόμενης ασθένειας, ένας από τους πρωταρχικούς στόχους των υπηρεσιών υγείας είναι η άμεση εύρεση των ατόμων που ήρθαν σε επαφή με έναν από τους φορείς ώστε να μειωθεί όσο το δυνατόν περισσότερο η μετάδοση σε περισσότερα άτομα. Δυστυχώς μία τέτοια διαδικασία εμφανίζει δυσκολίες. Πολλές φορές οι φορείς δεν θυμούνται τις ακριβείς τοποθεσίες στις οποίες βρέθηκαν με αποτέλεσμα να μην είναι δυνατή η εύρεση όλων των ατόμων με τους οποίους ήρθαν σε επαφή. 
\par
Επομένως ένα από τα μεγαλύτερα προβλήματα που εμφανίζονται είναι η αδυναμία των φορέων να ξαναφέρουν στη μνήμη τους όλα τα άτομα με τα οποία ήρθαν σε επαφή τις τελευταίες μέρες. Επιπλέον ένα ακόμη μεγάλο πρόβλημα που αντιμετωπίζουν οι υπηρεσίες υγείας είναι η αδυναμία να επικοινωνήσουν με τα άτομα που ήρθαν σε επαφή με έναν φορέα ώστε να τους προειδοποιήσουν. Ως εκ τούτου δημιουργείται η ανάγκη εύρεσης μια λύσης στα παραπάνω προβλήματα. 

\section{Στόχοι της εργασίας}
Η πτυχιακή επικεντρώνεται στην εύρεση τρόπων με τους οποίους οι νέες τεχνολογίες και συσκευές θα μπορούσαν να βοηθήσουν τις υπηρεσίες υγείας στην επίλυση των παραπάνω προβλημάτων. Εστιάζει στις δυνατότητες που προσφέρουν οι καινούριες κινητές συσκευές στην παρακολούθηση και καταγραφή των τοποθεσιών στις οποίες βρέθηκε ο χρήστης όπως επίσης και στην δυνατότητα της συνεχής πρόσβασης στο διαδίκτυο και ως εκ τούτου την ικανότητα άμεσης ενημέρωσης του.

\section{Δομή της εργασίας}
Η εργασία χωρίζεται σε επτά κεφάλαια. Αρχικά, στο επόμενο κεφάλαιο, θα υπάρξει μια θεωρητική ανάλυση των διάφορων εννοιών και τεχνολογιών που θα χρησιμοποιήθηκαν για την ανάπτυξη της πτυχιακής εργασίας. Έπειτα, θα παρουσιαστούν οι περιπτώσεις χρήσης του συστήματος. Αμέσως μετά, θα υπάρξει μια πλήρης περιγραφή του συστήματος στα διάφορα μέρη τα οποία το απαρτίζουν. Κατόπιν, θα παρουσιαστεί ο τρόπος υλοποίησης των διάφορων επιπέδων, καθώς επίσης θα αναλυθούν τα επιμέρους τεχνικά κομμάτια. Στη συνέχεια, θα παρουσιάσουμε τους τρόπους με τους οποίους έγινε ο έλεγχος ποιότητας του έργου. Τέλος, θα παρουσιαστούν τα διάφορα συμπεράσματα στα οποία καταλήξαμε καθώς και οι προτάσεις εξέλιξης του συστήματος.