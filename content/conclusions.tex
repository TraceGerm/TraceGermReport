\chapter{Επίλογος}
Ο σχεδιασμός και η υλοποίηση ενός τέτοιου συστήματος αποδείχθηκε ένα πολυσύνθετο έργο το οποίο προσέφερε εμπειρία και γνώση σε έναν μεγάλο αριθμό τεχνολογιών \citep{pragmatic}. Παρακάτω θα παρουσιάσουμε τα συμπεράσματα τα οποία προήλθαν για όλα τα επιμέρους κομμάτια του συστήματος μας και των τεχνολογιών οι οποίες χρησιμοποιήθηκαν.


\section{Εφαρμογή κινητών συσκευών}
Βλέποντας πίσω στην επιλογή που έγινε για τη χρήση των Ionic και Cordova frameworks για την δημιουργία της εφαρμογής κινητών συσκευών θα τολμούσαμε να πούμε ότι εμπεριείχε ένα μεγάλο ποσοστό ρίσκου το οποίο τελικά αποδείχτηκε στο μεγαλύτερο βαθμό επιτυχές. Θα μπορούσαμε να πούμε πως, οι παραπάνω τεχνολογίες, έχουν ωριμάσει σε τέτοιο βαθμό όπου ένας χρήστης δεν θα μπορούσε να δει καμία διαφορά σε επιδόσεις μεταξύ αυτών και των native εφαρμογών.

\par
Παρόλα αυτά εδώ πρέπει να αναφέρουμε πως εμφανίστηκαν δυσκολίες στο δρόμο. Η κυριότερη δυσκολία η οποία αντιμετωπίσθηκε ήταν η υπηρεσία καταγραφής των τοποθεσιών. Καθώς είναι ένα ιδιαίτερα δύσκολο κομμάτι ανεξαρτήτως πλατφόρμας κινητών συσκευών, η εύρεση μια βιβλιοθήκης η οποία να υποστηρίζει διάφορες πλατφόρμες εξίσου καλά ήταν ιδιαίτερη πρόκληση. 
\par
Ολοκληρώνοντας θα λέγαμε πως οι παραπάνω τεχνολογίες είναι έγκυρες τεχνολογίες για τη δημιουργία τέτοιων εφαρμογών παρόλα αυτά πάντα υπάρχει χώρος για βελτίωση. Νέες βιβλιοθήκες κάνουν την εμφάνιση τους καθημερινά και προσπαθούν να επιλύσουν θέματα όπως το παραπάνω.

\section{Διαδικτυακή εφαρμογή}
Η χρήση του AngularJS framework για τη δημιουργία της διαδικτυακής εφαρμογής έδειξε την τρομερή εξέλιξη των τεχνολογιών στο τομέα αυτό. Είδαμε πως πλέον η δημιουργία μιας διαδικτυακής εφαρμογής η οποία δίνει στο χρήστη την αίσθηση και τις επιδόσεις μια εφαρμογής υπολογιστών είναι ζωτικής σημασίας. Το AngularJS framework είναι ένα από τις πιο διαδεδομένες τεχνολογίες και πιο ώριμες στο τομέα αυτό. 

\section{Εξυπηρετητής}
Καθώς ο εξυπηρετητής μας έπρεπε να προσφέρει υπηρεσίες σε πολλαπλούς πελάτες η κατασκευή του ήταν ιδιαίτερα δύσκολη από την αρχή. Με τη βοήθεια του Spring framework καθώς και άλλων εργαλείων καταφέραμε να δημιουργήσουμε έναν εξυπηρετητή ο οποίος ανταποκρίνεται στα πρότυπα της αγοράς εργασίας. 

\section{Βάση δεδομένων}
Ένα από τα πιο ιδιαίτερα κομμάτια του συστήματος μας ήταν η βάση δεδομένων. Η ανάγκη αποθήκευσης γεωγραφικών δεδομένων, έφερε την ανάγκη εύρεσης εργαλείων τα οποία καθιστούν δυνατή μια τέτοια διαδικασία. Ύστερα από μελέτη διαφόρων λύσεων, τελικά είδαμε πως η βάση δεδομένων Postgress και η επέκταση PostGIS, ήταν η καλύτερη επιλογή. 

\par
Μέσα από την παραπάνω επιλογή έγινε η πρώτη επαφή με τεχνολογίες αυτού του είδους και παρουσιάστηκε για πρώτη φορά ο μεγάλος βαθμός δυσκολίας και η ιδιαιτερότητα του χώρο των γεωγραφικών δεδομένων. 

\section{Συμπεράσματα}
Βλέποντας το σύστημα μετά την ολοκλήρωση του, παρατηρούμε ότι η δημιουργία του χωρίς την χρήση των προαναφερθέντων framework θα ήταν ιδιαίτερα δύσκολη έως και αδύνατη σε ένα λογικό χρονικό διάστημα. Τα διάφορα προβλήματα τα οποία εμφανίστηκαν επιλύθηκαν μετά από προσωπική έρευνα καθώς και μέσα από την καθοδήγηση του επιβλέποντα καθηγητή. Τέλος η χρήση πολλαπλών framework καθ’ όλη τη διάρκεια της δημιουργίας του συστήματός, μας προσέφερε πολύτιμη εμπειρία για την επαγγελματική εξέλιξη στο χώρο.

\section{Προτάσεις εξέλιξης}
Το σύστημα υλοποιεί στο μεγαλύτερο βαθμό τους αρχικούς στόχους της πτυχιακής εργασίας. Παρ’ όλα αυτά σίγουρα επιδέχεται βελτίωσης και φυσικά περαιτέρω εξέλιξης. Ένα βασικό μέλος του συστήματος το οποίο θα μπορούσε να βελτιωθεί είναι το σύστημα καταγραφής τοποθεσιών στις εφαρμογές κινητών συσκευών. Θα μπορούσε να δοθεί περαιτέρω βαρύτητα στην απόδοση του συστήματος καθώς και στη κατανάλωση ενέργειας από τη μπαταρία της συσκευής. 

\par
Επιπλέον θα μπορούσε να προσφερθεί η δυνατότητα πρόβλεψης της πορείας μίας ασθένειας μέσω εξόρυξης δεδομένων από το σύστημα όπως επίσης και τη χρήση τεχνητής νοημοσύνης. Βέβαια η παραπάνω πρόταση θεωρείται άκρως ιδιαίτερη και ο βαθμός δυσκολίας της την καθιστά ιδιαίτερα δύσκολη να υλοποιηθεί ως πτυχιακή εργασία.

Ο πηγαίος κώδικας όπως και οποιαδήποτε πρόσθετη πληροφορία υπάρχει στην διεύθυνση
https://github.com/TraceGerm.
